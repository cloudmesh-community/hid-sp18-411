% status: 70
% chapter: Distributed Computing

\title{Apache Flink}


\author{Venkatesh Aditya Kaveripakam}
\affiliation{%
\institution{Indiana University
Bloomington} \city{Bloomington} \state{Indiana} \postcode{47408} \country{USA}
}
\email{vekave@iu.edu}

% The default list of authors is too long for headers}
\renewcommand{\shortauthors}{Venkatesh}

\keywords{hid-sp18-411, Flink, Stream Processing, Distributed Computing, Analytics }
%use up to 5 keywords

\maketitle

\section{Introduction}
Distributed data processing platforms for cloud computing are important tools
for large-scale data analytics. Hadoop MapReduce is the most prominent
distributed data processing technology since its inception in 2006. Hadoop
dominated the field of Big Data processing for over a decade, however the
implementation complexity and challenging programming model has led to the
development of advanced dataflow-oriented platforms. Apache Spark and Flink are
the modern technologies in the field of distributed data processing that
overcome the implementation complexity and performance overhead of Hadoop
systems. They are designed to satisfy real-time data processing requirements to
improve digital services in this generation of continuously evolving extremely
large-datasets. In this paper we discuss Apache Flink as a stream data
processing engine, its ability to process and yield continuous data analytics
results. We compare the pros and cons of Apache Flink with its closest
competitor Apache Spark.
\section{Data-sets and Processing methodologies}
In the space of Big Data processing it is first important to understand the type
of data sets and the processing models associated with them. There are 2 types
of datasets and 2 types of execution models that can be adopted by a distributed
data processing engine.
\begin{enumerate}
    \item \textbf{Unbounded dataset - } These are the data that are being
    generated continuously and in a limitless fashion. Unbounded data make up
    real-time data. They are also called as infinite data. Some examples this
    kind of data include user transaction data, sensor data, log data etc.
    \item \textbf{Bounded dataset - } These are the finite unchanging data whose
    complete information is well known. Quarterly marketing data of a company is
    one form of bounded data that is known and fixed during the time of analysis.
\end{enumerate}
\begin{enumerate}
    \item \textbf{Streaming Execution - } Processing that executes continuously
    as long as data is being produced. Commonly stream processing connects to
    external data sources, enabling applications to integrate certain data into
    the application flow.
    \item \textbf{Batch Execution - } Processing of data in batches and timely
    manner rather than continuous integration of data to the processing
    applications.
\end{enumerate} ~\cite{Apache-Flink}
Either type of datasets can be processed by using Streaming or Batch execution
model. Apache Flink however stands out in its ability to perform Streaming
execution on unbounded datasets. While Hadoop is a pure batch processing
framework and Apache spark support stream execution(micro batch)  execution,
Apache Flink is considered to be a true streaming engine.
\section{Apache Flink Features}
Flink is an open source Real Time Data Analytics (RTDA) framework that was
specifically designed to tackle the performance and implementation drawbacks of
Hadoop and Apache Spark. Flink uses a controlled cyclic dependency graph during
runtime for its data flow management. It is an operator-based streaming model.
A continuous flow operator is one that processes data when it arrives, without
any delay in collecting the data or processing the data.~\cite{[2]}
One of the biggest advantages of Flink is its ability to provide exactly-once
semantics for stateful computations. ‘Stateful’ applications maintain a summary
of data that has been processed over time, and Flink’s checkpointing mechanism
ensures exactly-once semantics for an application’s state in the event of a
failure. This also enhances the fault tolerance capacity of the Flink system,
where in the trade off between reliability and latency is minimal and there is
also zero data loss.
The combination of high throughput and low latency of Flink makes it one of the
fastest distributed data processing systems. Sometimes being able to process
data at 10 times faster speed than Apache Storm.


\section{conclusion}

\begin{acks}

    The authors would like to thank Dr.~Gregor~von~Laszewski for his
    support and suggestions to write this paper.

\end{acks}

\bibliographystyle{ACM-Reference-Format}
\bibliography{report}